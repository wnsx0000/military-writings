\part{\textit{\underline{기본행렬연산과 연립일차방정식}}}
\part*{1. 기본행렬연산과 기본행렬}

\section*{1. 기본연산(elementary operation)}

\subsubsection*{1) 정의\\}
\begin{DEF}
$m \times n$ 행렬 $A$에 대해서 $A$의 행[열]에 대한 아래의 세 연산을 기본행[열]연산(elementary row[column] operation)이라 함.

\begin{enumerate}
    \item $A$의 두 행[열]을 교환하는 것.
    \item $A$의 한 행[열]에 영이 아닌 스칼라를 곱하는 것.
    \item $A$의 한 행[열]에 다른 행[열]의 스칼라 배를 더하는 것.
\end{enumerate}

행연산(row operation)과 열연산(column operation)을 통틀어 기본연산(elementary operation)이라 함.

기본연산의 1, 2, 3을 각각 1형(type), 2형, 3형이라 함.
\end{DEF}

\section*{2. 기본행렬(elementary matrix)}

\subsubsection*{1) 정의\\}
\begin{DEF}
$n \times n$ 기본행렬(elementary matrix)은 항등행렬 $I_n$에 기본연산을 한 번 적용하여 얻은 행렬임.

$I_n$에 1형, 2형, 3형 연산을 하여 얻은 행렬을 각각 1형, 2형, 3형이라고 함.
\end{DEF}

\subsubsection*{2) 기본연산의 적용}
Theorem 3.1에 의하면, 기본연산을 적용하는 것은 그 행렬에 적절한 기본행렬을 곱하는 것과 같음.

이 덕분에 기본연산을 적용하는 것을 수식적으로 나타낼 수 있음.


\section*{3. 관련 정리}
\subsubsection*{1) 기본연산과 기본행렬의 곱}
\textbf{Theorem 3.1}\, 행렬 $A \in M_{m \times n}(F)$에 기본행[열]연산을 하여 행렬 $B$를 얻었다면, $B=EA[B=AE]$가 되는 $m \times m[n \times n]$ 기본행렬 $E$가 존재함. 이때, $A$에서 $B$를 얻은 기본행[열]연산을 $I_m[I_n]$에 똑같이 적용하면 행렬 $E$가 됨. 역으로 $E$가 $m \times m[n \times n]$ 기본행렬일 때, $I_m[I_n]$에서 $E$를 얻은 기본행[열]연산을 $A$에 똑같이 적용하면 $EA[AE]$가 됨.

즉, 기본행렬을 곱하는 것은 그 기본행렬에 해당하는 기본연산을 적용하는 것과 같음.

\subsubsection*{2) 기본행렬의 가역성}
\textbf{Theorem 3.2}\, 기본행렬은 가역임. 그 역행렬은 같은 종류의 기본행렬임.

\begin{proof}
기본행렬을 만든 연산을 거꾸로 수행하면(해당하는 기본행렬 곱하면) 항등행렬이 나오므로 기본행렬은 가역임.\\
\end{proof}


\newpage


\part*{2. 행렬의 랭크}

행렬의 랭크로 임의의 $n \times n$ 행렬이 가역인지를 알 수 있음.

행렬이 가역인지 판단하면 해당 선형변환의 가역성 또한 알 수 있음.

\section*{1. 행렬의 랭크}
\subsubsection*{1) 정의\\}
\begin{DEF}
행렬 $A \in M_{m \times n}(F)$에 대해서 $A$의 랭크(rank)는 선형변환 $L_A:F^n \rightarrow F^m$의 랭크로 정의하고 $rank(A)$라 표기함.
\end{DEF}

\subsubsection*{2) 행렬의 랭크와 가역성}
\textbf{Theorem}\, $n \times n$ 행렬이 가역이기 위한 필요충분조건은 행렬의 랭크가 $n$인 것임. 행렬이 가역이 아니기 위한 필요충분조건은 행렬의 랭크가 $n$보다 작은 것임.

그렇기 때문에, 정사각행렬의 행렬의 랭크를 구하면 그 행렬이 가역인지를 판단할 수 있음.

\begin{proof}
1. $n \times n$ 행렬 $A$가 가역 $\rightarrow$ 행렬 $A$의 랭크가 $n$.
Theorem 2.18의 Corollary에 의해, $A$가 가역이면 $L_A$도 가역임. 즉 $dim(F^n)=rank(L_A)=n$임. $L_A$의 랭크가 행렬 $A$의 랭크이므로 성립.

2. 행렬 $A$의 랭크가 $n$ $\rightarrow$ $n \times n$ 행렬 $A$가 가역.
$rank(L_A)=dim(F^n)$이고, $L_A:F^n \rightarrow F^n$으로 정의역과 공역의 차원이 같으므로 행렬 $A$는 가역임.
\end{proof}

\subsubsection*{3) 행렬의 랭크와 선형변환의 랭크}
Theorem 3.3에 의하면, 선형변환의 랭크와 그 행렬표현의 랭크는 동일함.

그러므로, 선형변환의 랭크를 찾는 문제는 그 행렬표현의 랭크를 찾는 문제와 귀결됨.\\


\section*{2. 행렬의 랭크 구하기}
\subsubsection*{1) 행렬의 랭크를 보존하는 연산}
Theorem 3.4에 의하면, 가역행렬의 곱은 행렬의 랭크를 보존하는 연산임.

Theorem 3.4의 Corollary에 의해, 행렬의 기본연산은 랭크를 보존함.\\
즉, 행렬에 기본연산을 적용하여(기본행렬을 곱하여) 랭크를 구하기 더 쉬운 형태로 바꿀 수 있음.

\subsubsection*{2) 행렬의 랭크 구하기}
정리하면 행렬의 랭크는 아래의 방법으로 구할 수 있음.
\begin{enumerate}
    \item 기본연산으로 행렬을 정리함. (Theorem 3.4)
    \item 일차독립인 열의 개수를 확인함. (Theorem 3.5, Theorem 3.6, Theorem 3.6 Corollary 2)
\end{enumerate}

즉, 행렬의 일차독립인 행 또는 열이 보일 때까지 기본연산을 적용하여 간단히 만드는 것.\\


\newpage


\section*{4. 관련 정리}
\subsubsection*{1) 행렬의 랭크과 선형변환의 랭크}
\textbf{Theorem 3.3}\, 유한차원 벡터공간 사이에서 정의된 선형변환 $T:V \rightarrow W$와 $V,W$ 각각의 순서기저 $\beta, \gamma$에 대해서 $rank(T)=rank([T]^{\gamma}_{\beta})$임.

즉, 선형변환의 랭크과 그 행렬표현의 랭크는 동일함.

\subsubsection*{2) 행렬의 랭크를 보존하는 연산}
\textbf{Theorem 3.4}\, $m \times n$ 행렬 $A$, $m \times m$ 가역행렬 $P$, $n \times n$ 가역행렬 $Q$에 대해서 아래가 성립함.

\begin{enumerate}
    \item $rank(AQ)=rank(A)$
    \item $rank(PA)=rank(A)$
    \item $rank(PAQ)=rank(A)$
\end{enumerate}

즉, 가역행렬의 곱은 행렬의 랭크를 보존하는 연산임.

\textbf{Corollary}\, 행렬의 기본행연산과 기본열연산은 랭크를 보존함.

\begin{proof}
기본연산은 기본행렬을 곱하는 것인데, 기본행렬은 정사각행렬인 가역행렬이므로 행렬의 랭크를 보존함.
\end{proof}

\subsubsection*{3) 행렬의 랭크}
\textbf{Theorem 3.5}\, 임의의 행렬의 랭크는 일차독립인 열의 최대 개수와 같음. 즉, 행렬의 랭크는 그 열에 의해 생성된 부분공간의 차원임.

즉, 행렬의 각 열을 하나의 벡터로 생각했을 때, 일차독립인 열들의 집합을 만들면 그 개수가 곧 랭크임.

행렬의 열은 곧 기저를 보낸 것을 의미하는데, 상공간 생성 방법을 생각해 보면 이 정리는 매우 당연함.


\subsubsection*{4) 행렬의 랭크를 구하기 위한 구체적 방법}
\textbf{Theorem 3.6}\, 랭크가 $r$인 $m \times n$ 행렬 $A$를 생각하자. $r \leq m,\,r \leq n$이 성립하고 기본행연산과 기본열연산을 유한 번 사용하여 $A$를 아래와 같은 꼴로 바꿀 수 있음.
\[
D=
\begin{pmatrix}
I_r & O_1\\
O_2 & O_3
\end{pmatrix}
\]

이때, $i \leq r$이면 $D_{ii}=1$, 그렇지 않으면 $D_{ij}=0$이고 $O_1,\,O_2,\,O_3$은 영행렬임.

즉, 행렬에 기본연산을 유한 번 사용해 왼쪽 위가 $I_r$이고 나머지는 0인 행렬로 만들 수 있음. 이 꼴로 만들면 일차독립인지를 확인하는 것이 굉장히 간단해짐.\footnote{물론 정확히 이렇게 만들 필요는 없고, 랭크를 구할 수 있을 정도까지만 연산을 하면 됨.}

증명은 프리드버그 p.179에 있지만 굳이 정리하지 않겠음.


\newpage


\textbf{Corollary 1}\, Theorem 3.5의 행렬 $A$에 대해서, $D=BAC$를 만족하는 $m \times m$ 가역행렬 $B$, $n \times n$ 가역행렬 $C$가 존재함.

즉, 행렬에 기본행렬을 곱해 $D$로 만들 수 있다는 것.

\textbf{Corollary 2}\, $m \times n$ 행렬에 대해 아래가 성립함.
\begin{enumerate}
    \item $rank(A^t)=rank(A)$
    \item 임의의 행렬의 랭크는 일차독립인 행의 최대 개수와 같음. 행렬의 랭크는 그 행에 의해 생성된 부분공간의 차원임.
    \item 임의의 행렬의 행과 열은 차원이 같은 부분공간을 생성함. 각각의 차원은 행렬의 랭크와 같음.
\end{enumerate}

\begin{proof}
(1) Corollary 1에 의하면 $D=BAC$인데, $D^t=(BAC)^t=C^{t}A^{t}B^{t}$임. $C^{t}$, $B^{t}$는 가역이므로 $rank(C^{t}A^{t}B^{t})=rank(A^{t})=rank(D^t)$인데, $rank(D^t)=rank(D)=rank(A)$임.

(2) 전치해 보면 확인할 수 있음.

(3) Theorem 3.5, Theorem 3.6의 Corollary 2 (1), (2)를 보면 알 수 있음.
\end{proof}

\textbf{Corollary 3}\, 모든 가역행렬은 기본행렬의 곱으로 나타남.

\begin{proof}
$n \times n$ 가역행렬 $A$의 랭크는 $n$임. $D=I_n=BAC$임. $B$와 $C$는 각각 $B=E_{p}E_{p-1} \cdots E_{1}$, $C=G_{1}G_{2} \cdots G_{q}$를 만족하는 기본행렬 $E_{i},\,G_{i}$가 존재함. 정리하면 아래와 같음.

\[
A=B^{-1}I_{n}C^{-1}=B^{-1}C^{-1}=E_{1}E_{2} \cdots E_{p}G_{q}G_{q-1} \cdots G_{1}
\]

즉, 행렬 $A$는 기본행렬의 곱으로 나타낼 수 있음.
\end{proof}

\subsubsection*{5) 합성과 행렬 곱에 따른 랭크}
\textbf{Theorem 3.7}\, 유한차원 벡터공간 $V,W,Z$ 사이에 정의된 선형변환 $T:V \rightarrow W$, $U:W \rightarrow Z$와 행렬 곱 $AB$를 정의하는 두 행렬 $A,\,B$에 대해서 아래가 성립함.

\begin{enumerate}
    \item $rank(UT) \leq rank(U)$
    \item $rank(UT) \leq rank(T)$
    \item $rank(AB) \leq rank(A)$
    \item $rank(AB) \leq rank(B)$
\end{enumerate}

즉, 선형변환의 합성 또는 행렬의 곱은 랭크를 더 커지게 할 수 없음.

\begin{proof}
(1) $R(UT)=(UT)(V)=U(T(V)) \subseteq U(W)=R(U)$임.

(3) (1)이 성립하므로 선형변환을 행렬표현으로 나타내면 성립함을 확인할 수 있음.

(4) (3)이 성립하므로 $rank(AB)=rank(B^{-1}A^{-1}) \leq rank(A^{-1})=rank(A)$으로 성립함.

(2) (4)가 성립하므로 행렬표현을 선형변환으로 나타내면 성립함을 확인할 수 있음.

\end{proof}


\newpage


\part*{3. 역행렬 구하기}

행렬의 랭크로 가역성을 확인했으면, 해당 행렬의 역행렬은 아래의 방법으로 구할 수 있음.

역행렬을 구하면 해당 선형변환의 역변환도 알아낼 수 있음.

\section*{1. 첨가행렬(augmented matrix)}
\subsubsection*{1) 정의\\}
\begin{DEF}
$m \times n$ 행렬 $A$와 $m \times p$ 행렬 $B$에 대해서 첨가행렬(augmented matrix) $(A|B)$는 $m \times (n+p)$ 행렬 $(A\,B)$임. 즉, 처음 $n$개 열은 $A$의 열이고, 그 다음 $p$개 열은 $B$의 열인 행렬임.
\end{DEF}

쉽게 말해, 행렬 $A$의 오른쪽에 $B$를 그대로 붙인 행렬을 첨가행렬 $(A|B)$라 하는 것.

\subsubsection*{2) 행렬과 첨가행렬의 곱}
$n$개의 행을 가지는 행렬 $A,B$와 $m \times n$ 행렬 $M$에 대해서 아래가 성립함.

\[
M(A|B)=(MA|MB)
\]

즉, 왼쪽에 곱한 행렬의 연산이 분배법칙처럼 각각 적용됨.


\section*{2. 기본행연산으로 역행렬 구하기}
\subsubsection*{1) 역행렬 구하기}
\textbf{Theorem}\, $A$가 $n \times n$ 가역행렬이면 행렬 $(A|I_n)$에 기본행연산을 유한 번 적용해서 $(I_n|A^{-1})$로 변형할 수 있음.

즉, $(A|I_n)$에 기본행연산을 하여 $(I_n|A^{-1})$을 만들 수 있다는 것.

\begin{proof}
$A^{-1}(A|I_n)=(A^{-1}A|A^{-1}I_n)=(I_n|A^{-1})$임. Theorem 3.6의 Corollary 3에 의해, $A^{-1}$는 정사각행렬이므로 기본행렬의 곱으로 나타낼 수 있음. 그런데 왼쪽에 곱해진 기본행렬은 기본행연산이므로, $(A|I_n)$에 기본행연산을 하여 $(I_n|A^{-1})$을 만들 수 있음.
\end{proof}

\subsubsection*{2) 변형이 되면 역행렬임}
\textbf{Theorem}\, $n \times n$ 가역행렬 $A$에 대해서, 첨가행렬 $(A|I_n)$에 기본행연산을 유한 번 적용하여 $(I_n|B)$로 변형할 수 있으면 $B=A^{-1}$임.

즉, 첨가행렬을 기본행연산으로 변형해서 일단 $(I_n|B)$꼴을 만들면 $B$가 역행렬임. (다른 이상한 행렬이 나오지 않음.)

\begin{proof}
행렬 $C=E_{p}E_{p-1} \cdots E_1$일 때, $C(A|I_n)=(CA|C)=(I_n|B)$임. $CA=I_n$, $C=B$이므로 $B=A_{-1}$임.
\end{proof}

\subsubsection*{3) 가역이 아닌 경우}
\textbf{Theorem}\, 가역인 아닌 $n \times n$ 행렬 $A$에 대해서, $(A|I_n)$에 기본행연산을 적용하여 $(I_n|B)$ 꼴로 변형을 시도하면 성공하지 못하고 앞쪽 $n$개 성분이 모두 0인 행을 가진 행렬을 얻게 됨.

\begin{proof}
$A$가 가역이 아니므로 $rank(A) < n$임. 유한 번의 기본연산으로 $(A|I_n)$을 $(I_n|B)$로 바꿀 수 있다고 가정하면, 유한 번의 기본연산으로 $A$을 $I_n$로 바꿀 수 있어야 하는데 기본연산은 랭크를 보존하므로 $rank(A)=rank(I_n)=n$으로 모순임. 즉, 유한 번의 기본연산으로 $(A|I_n)$을 $(I_n|B)$로 바꿀 수 없음.
\end{proof}


\newpage


\section*{3. 선형변환의 가역성/역함수 구하기}
\subsubsection*{1) 선형변환의 가역성}
선형변환의 가역성은 Theorem 2.5와 가역성의 정의로 알아낼 수 있지만, 특히 정사각행렬인 경우 행렬로 알아낼 수도 있음.

선형변환의 행렬표현이 $n \times n$ 행렬인 경우 랭크가 $n$인지를 확인하여 가역성을 확인할 수 있음.

\subsubsection*{2) 역변환 구하기}
선형변환 행렬표현의 가역성을 확인한 후 역행렬을 구하면, Theorem 2.18에 의해 $[T^{-1}]_{\gamma}^{\beta}=([T]_{\beta}^{\gamma})^{-1}$으로 해당 역행렬은 해당 선형변환 역변환의 행렬표현임.

역행렬에 임의의 벡터를 넣는 방식으로 선형변환을 알아낼 수 있음.

아래는 그 예시임. $T:P_2(R) \rightarrow P_2(R)$, $P_2(R)$의 표준 순서기저를 $\beta$라 함.

\[
[T^{-1}(a_0 + a_1x + a_2x^2)]_{\beta}=[T^{-1}]_{\beta}[(a_0 + a_1x + a_2x^2)]_{\beta}=
\begin{pmatrix}
a_0 - a_1\\
a_1 - 2a_2\\
a_2
\end{pmatrix}
\]

즉, $T^{-1}(a_0 + a_1x + a_2x^2)=(a_0 - a_1)+(a_1 - 2a_2)x + a_2x^2$임.


\newpage


\part*{4. 연립일차방정식 : 이론적 측면}

연립일차방정식이 나오면 두 가지 질문에 완벽히 답할 수 있어야 함.\\
1. 주어진 연립방정식에 해가 있는가? (Theorem 3.11)\\
2. 해가 있다면 모든 해(해집합)를 어떻게 구할 수 있는가? (Theorem 3.8, Theorem 3.9, Theorem 3.10)

\section*{1. 연립일차방정식(system of linear equations)}
\subsubsection*{1) 정의\\}
\begin{DEF}
아래의 형태를 체 $F$ 위 $n$개의 미지수와 $m$개의 일차방정식으로 이루어진 연립일차방정식(system of linear equations)이라 함.

\begin{gather*}
a_{11}x_{1}+a_{12}x_{2}+ \cdots + a_{1n}x_{n}=b_1 \\
\vdots \\
a_{m1}x_{1}+a_{m2}x_{2}+ \cdots + a_{mn}x_{n}=b_m
\end{gather*}

이때, $a_{ij}$와 $b_i$는 $F$의 스칼라이고, $x_i$는 $F$에서 값을 가지는 변수임.
\end{DEF}

\subsubsection*{2) 계수행렬(coefficient matrix)\\}
\begin{DEF}
아래의 $m \times n$ 행렬 $A$를 연립일차방정식 $(S)$의 계수행렬(coefficient matrix)이라 함.

\[
A=
\begin{pmatrix}
a_{11} & a_{12} & \cdots & a_{1n}\\
a_{21} & a_{22} & \cdots & a_{2n}\\
\vdots & \vdots & & \vdots \\
a_{m1} & a_{m2} & \cdots & a_{mn}
\end{pmatrix}
\]
\end{DEF}

이때, 아래와 같이 정의하면 연립일차방정식 $(S)$는 하나의 행렬 식(matrix equation) $Ax=b$로 나타낼 수 있음.

\[
x=
\begin{pmatrix}
x_1 \\
x_2 \\
\cdots \\
x_n
\end{pmatrix}
,\,b=
\begin{pmatrix}
b_1 \\
b_2 \\
\cdots \\
b_m
\end{pmatrix}
\]

\subsubsection*{3) 해집합(solution set)\\}
\begin{DEF}
$As=b$인 $n$순서쌍 $s$를 연립일차방정식 $(S)$의 해(solution)라 하고, 연립일차방정식 $(S)$가 가지는 모든 해들의 집합을 해집합(solution set)이라 함.

\[
s=
\begin{pmatrix}
s_1 \\
s_2 \\
\cdots \\
s_n
\end{pmatrix}
\in F^n
\]
\end{DEF}

해집합이 공집합이 아니면 이 연립일차방정식을 모순이 없다(consistent) 또는 해가 존재한다고 함.\\
해집합이 공집합이면 이 연립일차방정식을 모순이 있다(inconsistent) 또는 해가 존재하지 않는다고 함.

연립일차방정식은 해가 하나이거나, 해가 무한히 많거나, 해가 없음.


\newpage


\section*{2. 동차(homogeneous)/비동차(non-homogeneous)\footnote{연립일차방정식의 풀이는 동차 연립일차방정식부터 시작하여, 동차/비동차 연립일차방정식의 해집합을 부분공간으로 묘사하는 것임.}}
\subsubsection*{1) 정의\\}
\begin{DEF}
$n$개의 미지수와 $m$개의 일차방정식으로 이루어진 연립일차방정식 $Ax=b$는 $b=0$일 때, 동차(homogeneous)라 함. 동차가 아닌 연립방정식은 비동차(non-homogeneous)라 함.
\end{DEF}

임의의 동차 연립일차방정식에는 적어도 하나의 해가 있음. (영벡터)

\subsubsection*{2) 동차 연립일차방정식의 해집합 구하기}
동차 연립일차방정식 해집합의 기저 하나만 찾으면 그 해집합을 알 수 있음.

기저를 찾는 가장 간단한 방법은 해집합의 차원을 찾고, $nullity(L_A)$개의 해를 대충 맞춰서 찾아내는 것임.\footnote{대체정리의 Corollary 사용.}\\
기저를 찾는 구체적인 방법은 '5. 연립방정식 : 계산적 측면'에서 다룸.

Theorem 3.8에 의하면 동차 연립일차방정식의 해집합은 kernel(null space)로 부분공간이므로 기저가 존재함.

해집합의 차원을 쉽게 찾는 방법은 아래와 같음.
\begin{enumerate}
    \item 행렬의 열의 개수를 세면 그게 $dim(V)$임.
    \item 행렬의 랭크를 알아냄.
    \item 열의 개수와 랭크를 빼면 해집합의 차원임.
\end{enumerate}

\subsubsection*{3) 비동차 연립일차방정식의 해집합 구하기}
Theorem 3.9에 의하면, 비동차 연립일차방정식의 해집합은 대응하는 동차 연립일차방정식의 해집합으로 알아낼 수 있음.

$Ax=0$을 $Ax=b$에 대응하는 동차 연립일차방정식이라고 함.

\subsubsection*{4) 계수행렬이 가역인 연립일차방정식}
Theorem 3.10에 의해, 계수행렬이 가역인 연립일차방정식은 유일한 해를 간단히 구할 수 있음.

\subsubsection*{5) 연립일차방정식이 해를 가지는지 판별하기}
Theorem 3.11에 의해, $Ax=b$에서 $rank(A)=rank(A|b)$인지를 보면 해를 가지는지 판별할 수 있음.\\


\newpage


\section*{3. 관련 정리}
\subsubsection*{1) 동차 연립일차방정식의 해집합}
\textbf{Theorem 3.8}\, 체 $F$에서 $n$개의 미지수와 $m$개의 일차방정식으로 이루어진 연립일차방정식 $Ax=0$을 생각하자. 방정식 $Ax=0$의 해집합을 $K$라 할 때, $K=N(L_A)$임. 즉, $K$는 $F^n$의 부분공간이고 차원은 $n-rank(L_A)=n-rank(A)$임.

즉, 동차 연립일차방정식의 해집합은 $L_A$의 kernel(null space)이고, 그 차원은 $n-rank(A)$임.\\

\begin{proof}
$L_A$는 왼쪽에 $A$를 곱하는 선형변환이므로, $Ax=0$인 $x$는 kernel(space)에 속함. 차원은 차원정리로 생각할 수 있음.
\end{proof}

\textbf{Corollary}\, $m < n$이면 연립일차방정식 $Ax=0$은 영벡터가 아닌 해가 있음.

\begin{proof}
$Ax=0$의 해집합이 $N(L_A)$이므로, $Ax=0$이 영벡터가 아닌 해를 가진다는 것은 $N(L_A)$ 영벡터가 아닌 원소를 가진다는 것임. 즉, $nullity(L_A) \neq 0$이어야 함. $rank(A)=rank(L_A) \leq m$이므로, $nullity(L_A)=n-rank(L_A) \geq n-m > 0$으로 $nullity(L_A) \neq 0$임.
\end{proof}


\subsubsection*{2) 비동차 연립일차방정식의 해집합}
\textbf{Theorem 3.9}\, 모순이 없는 연립일차방정식 $Ax=b$의 해집합을 $K$, 대응하는 연립일차방정식 $Ax=0$의 해집합을 $K_H$라 하자. $Ax=b$의 임의의 해를 $s$라 하면 아래가 성립함.

\[
K=\{s\}+K_H=\{s+k | k \in K_H\}
\]

즉, $Ax=b$의 임의의 해 하나를 고정하고, 그 해와 $Ax=0$의 해집합의 원소들을 각각 더한 벡터들이 $Ax=b$의 해집합이라는 것.

\begin{proof}
1. $K \subseteq \{s\}+K_H$
$w,s \in K$에 대해서, $Aw=b,As=b,A(w-s)=0$이므로 $k=w-s \in K_H$임. 즉, $w=s+k \in \{s\}+K_H$이고 $K \subseteq \{s\}+K_H$임,

2. $\{s\}+K_H \subseteq K$
$w \in \{s\}+K_H$, $k \in K_H$에 대해서, $w=s+k$, $Aw=As+Ak=b$이므로 $w \in K$임. 즉, $\{s\}+K_H \subseteq K$임.

$K \subseteq \{s\}+K_H$이고 $\{s\}+K_H \subseteq K$이므로  $K=\{s\}+K_H$임.
\end{proof}

\subsubsection*{3) 행렬의 가역성과 유일한 해}
\textbf{Theorem 3.10}\, $n$개의 미지수와 $n$개의 일차방정식으로 이루어진 연립일차방정식 $Ax=b$를 생각하자. 행렬 $A$가 가역이면 이 연립일차방정식은 유일한 해 $A^{-1}b$가 있음. 역으로, 이 방정식의 해가 유일하면 행렬 $A$는 가역임.

즉, $n \times n$ 행렬이 가역이면 연립일차방정식이 유일한 해($A^{-1}b$)를 가지고, 연립일차방정식이 유일한 해를 가지면 $n \times n$ 행렬이 가역임.

\begin{proof}
1. $n \times n$ 행렬이 가역이면 연립일차방정식이 유일한 해($A^{-1}b$)를 가짐.
$AA^{-1}b=b$. $s \in K$인 $k$가 존재한다고 가정하자. $A^{-1}As=A^{-1}b,s=A^{-1}b$이므로 연립일차방정식이 유일한 해를 가짐.

2. 연립일차방정식이 유일한 해를 가지면 $n \times n$ 행렬이 가역임.
$As=b$인 $s \in K$가 유일하다고 하자. Theorem 3.9에 의해 $\{s\}=\{s\}+K_H$이므로 $K_H=\{0\}$임. 즉, $n=rank(L_A)+nullity(L_A)=rank(A)$이므로 $A$는 가역임.
\end{proof}

\subsubsection*{4) 연립일차방정식이 해를 가지는지 판별하기}
\textbf{Theorem 3.11}\, 연립일차방정식 $Ax=b$에 모순이 없기 위한 필요충분조건은 $rank(A)=rank(A|b)$인 것임.

\begin{proof}
모순이 없음 $\Leftrightarrow$ $Ax=b$가 해를 가짐 $\Leftrightarrow$ $b \in R(L_A)$임. $\Leftrightarrow$ $R(L_A)=span(\{a_1, \cdots ,a_n\})$($a_i$는 $A$의 $i$번째 열)이므로, $b \in span(\{a_1, \cdots ,a_n\})$ $\Leftrightarrow$ $span(\{a_1, \cdots ,a_n\})=span(\{a_1, \cdots ,a_n,b\})$. 즉, $b$가 $a_i$의 일차결합으로 표현됨. $\Leftrightarrow$ $dim(span(\{a_1, \cdots ,a_n\}))=dim(span(\{a_1, \cdots ,a_n,b\}))$ $\Leftrightarrow$ $rank(A)=rank(A|b)$임.\\
\end{proof}


\part*{5. 연립일차방정식 : 계산적 측면}

기본행연산을 사용하여 연립방정식의 모든 해를 찾을 수 있음.

\section*{1. 동치(equivalent)}
\subsubsection*{1) 정의\\}
\begin{DEF}
두 연립일차방정식의 해집합이 서로 같을 때, 두 연립일차방정식은 동치(equivalent)라 함.
\end{DEF}

어떤 연립일차방정식의 해집합을 구할 때, 동치인 더 쉬운 연립일차방정식으로 바꾸어 구하는 것이 더 쉬움.

\subsubsection*{2) 동치인 연립일차방정식으로 전환하기}
Theorem 3.13의 Corollary에 의하면, 연립일차방정식 $Ax=b$에 대해 $(A|b)$에 기본행연산을 적용한 $(A^{\prime}|b^{\prime})$의 $A^{\prime}x=b^{\prime}$가 $Ax=b$와 동치임.

즉, $(A|b)$에 기본행연산을 적용하여 더 쉬운 연립일차방정식으로 바꾸어 해를 구할 수 있음.

\section*{2. 행간소사다리꼴(기약행사다리꼴, reduced row echelon form}
\subsubsection*{1) 정의\\}
\begin{DEF}
아래의 세 조건을 만족하는 행렬을 행간소사다리꼴 또는 기약행사다리꼴(reduced row echelon form)이라고 함.

\begin{enumerate}
    \item 0이 아닌 성분을 가지는 행은 모든 성분이 0인 행보다 위에 위치함.
    \item 각 행의 처음으로 0이 아닌 성분은 그 성분을 포함하는 열에서 유일하게 0이 아닌 성분임.
    \item 각 행에서 처음으로 0이 아닌 성분은 1이고, 이전 행의 처음으로 0이 아닌 성분보다 오른쪽에 위치함.
\end{enumerate}
\end{DEF}

$(A|b)$꼴 연립일차방정식을 행간소사다리꼴로 바꾸면 계산이 굉장히 간단해짐. 또한, 행간소사다리꼴을 이용하여 해집합과 해의 존재 유무를 알아낼 수 있음. 즉, 그 연립일차방정식의 모든 것을 알게됨.

Theorem 3.16의 Corollary에 의하면 어떤 행렬에 대해서 그 행렬의 행간소사다리꼴은 유일함.


\newpage


\subsubsection*{2) 행간사다리꼴로 해집합 구하기}
행간사다리꼴로 원래 연립일차방정식의 해집합을 구하는 방법은 아래와 같음.

1. 모든 성분이 0인 행은 무시함.

2. 각 행에 대응하는 일차방정식에서, 가장 왼쪽에 있는 변수들에 매개변수를 부여함.\footnote{$x_1=t_1, x+3=T_2$등으로 설정함.}

3. 나머지 변수들을 매개변수를 부여한 변수들로 나타냄.

4. 매개변수로 표현한 변수들로 해를 나타내고, 매개변수에 대해서 정리함. 이것이 원래 연립일차방정식의 임의의 해임. 이때, 매개변수에 곱해져 있는 행렬들의 집합은 동차 연립일차방정식 해집합의 기저임. 또한 매개변수가 곱해져 있지 않은 행렬은 비동차 연립일차방정식의 한 해임. 이를 특수해(particular solution)이라고 함.

아래는 이에 대한 예시임.

\[
\begin{pmatrix}
x_1\\
x_2\\
x_3\\
x_4\\
x_5
\end{pmatrix}
=
\begin{pmatrix}
-2t_1+2t_2+3\\
t_1-t_2+1\\
t_1\\
2t_2+2\\
t_2
\end{pmatrix}
=
\begin{pmatrix}
3\\
1\\
0\\
2\\
0
\end{pmatrix}
+t_1
\begin{pmatrix}
-2\\
1\\
1\\
0\\
0
\end{pmatrix}
+t_2
\begin{pmatrix}
2\\
-1\\
0\\
2\\
1
\end{pmatrix}
\]

정리하면, 연립일차방정식의 첨가행렬을 행간소사다리꼴로 바꾸면 아래의 두 가지를 알아낼 수 있음.\\
1. 처음 연립일차방정식($Ax=b$)의 특수해(한 해).\\
2. 처음 연립일차방정식에 대응하는 동차 연립일차방정식의 해집합의 기저.

즉, 처음 연립일차방정식의 해집합을 알아낼 수 있음.

\subsubsection*{3) 행간소사다리꼴로 해의 존재 유무 판정하기}
\textbf{Theorem}\, 연립일차방정식의 해가 존재하기 위한 필요충분조건은 첨가행렬을 정리하여 행간소사다리꼴을 만들 때, 0이 아닌 유일한 성분이 마지막 열에 있는 행이 존재하지 않는 것임.

즉, 행간소사다리꼴의 각 행에 대해서, 처음으로 나온 1이 마지막 열(맨 오른쪽)에 존재하면 해가 존재하지 않음.\\


\section*{3. 가우스 소거법(Gaussian elimination)}
\subsubsection*{1) 정의\\}
\begin{DEF}
1. 위에서 아래로\\
위에서 아래로 내려가면서 첨가행렬을 변형하여 각 행의 최초로 0이 아닌 성분은 1이고, 이 성분이 이전 행의 최초로 0이 아닌 성분보다 오른쪽 열에 위치하는 행렬(상삼각행렬)로 만듦. (정의 1,3 만족)

2. 아래에서 위로\\
아래에서 위로 올라가면서 행렬(상삼각행렬)을 변형하여 각 행의 최초로 0이 아닌 성분이 이 성분을 포함하는 열에서 유일하게 0이 아닌 성분인 행간소사다리꼴로 만듦. (정의 2 만족)
\end{DEF}

Theorem 3.14에 의하면, 가우스 소거법은 첨가행렬을 행간소사다리꼴로 만듦.

행렬을 행간소사다리꼴로 변환할 때, 산술 연산을 가장 적게 하는 방법이 가우스 소거법임.

모든 행렬은 가우스 소거법을 사용하여 행간소사다리꼴로 바꿀 수 있음.\\


\newpage


\section*{4. 일반해}
\subsubsection*{1) 정의\\}


\section*{5. 행간사다리꼴에 대한 해석}
\subsubsection*{1) }


\section*{6. 관련 정리}
\subsubsection*{1) 동치인 두 연립일차방정식}
\textbf{Theorem 3.13}\, $n$개의 미지수와 $m$개의 일차방정식으로 이루어진 연립일차방정식 $Ax=b$와 $m \times m$ 갸역행렬 $C$에 대해서, 아래의 두 연립일차방정식은 동치임.

\[
Ax=b,\,(CA)x=Cb
\]

\begin{proof}
$Ax=b$의 해집합을 $K$, $(CA)x=Cb$의 해집합을 $K^{\prime}$이라 하자.

1. $K \subseteq K^{\prime}$\\
$w \in K$가 있음. $Aw=b$이므로, $(CA)w=Cb$, $w \in K^{\prime}$임. 즉, $K \subseteq K^{\prime}$임.

2. $K^{\prime} \subseteq K$\\
$w \in K^{\prime}$가 있음. $(CA)w=Cb, C^{-1}(CA)w=C^{-1}Cb, Aw=b, w \in K$임. 즉, $K^{\prime} \subseteq K$임.

즉, $K=K^{-1}$임.
\end{proof}

\textbf{Corollary}\, $n$개의 미지수와 $m$개의 일차방정식으로 이루어진 연립일차방정식 $Ax=b$가 있음. $(A|b)$에 기본행연산을 유한 번 적용하여 얻은 $(A^{\prime}|b^{\prime})$은 처음 주어진 연립일차방정식과 동치임.

\subsubsection*{2) 가우스 소거법}
\textbf{Theorem 3.14}\, 가우스 소거법은 임의의 행렬을 행간소사다리꼴로 바꾸어 줌.







\newpage
