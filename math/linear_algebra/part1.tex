\part{\textit{\underline{벡터공간}}}

1장은 벡터공간에 대한 이야기임.

\part*{1. 벡터공간}

\section*{1. 벡터공간(vector space)}

\subsubsection*{1) 정의\\}
\begin{DEF}
체 $F$에서의 벡터공간(vector space) 또는 선형공간(linear space) $V$는 다음 8가지 조건을 만족시키는 두 연산, 합과 스칼라 곱을 가지는 집합임.

\begin{itemize}
\item 합(sum)은 $V$의 두 원소 $x, y$에 대하여 유일한 원소 $x+y \in V$를 대응하는 연산임.
\item 스칼라 곱(scalar multiplication)은 체 $F$의 원소 $a$와 벡터공간 $V$의 원소 $x$마다 유일한 원소 $ax \in V$를 대응하는 연산이다. 이때 $ax$는 $a$와 $x$의 스칼라 곱(product)이라 함.
\end{itemize}

(VS1) 모든 $x,y \in V$에 대하여 $x+y=y+x$임. (덧셈의 교환법칙)\\
(VS2) 모든 $x,y,z \in V$에 대하여 $(x+y)+z=x+(y+z)$임. (덧셈의 결합법칙)\\
(VS3) 모든 $x \in V$에 대하여 $x+0=x$인 $0 \in V$가 존재함. (덧셈에 대한 항등원, 즉 영벡터 존재)\\
(VS4) 각 $x \in V$마다 $x+y=0$인 $y \in V$가 존재함. (덧셈에 대한 역원, 즉 역벡터 존재)\\
(VS5) 각 $x \in V$에 대하여 $1x=x$임. (스칼라 곱에 대한 항등원 존재)\\
(VS6) 모든 $a,b \in F$와 모든 $x \in V$에 대하여 $(ab)x=a(bx)$임. (스칼라 곱에 대한 결합법칙)\\
(VS7) 모든 $a \in F$와 모든 $x,y \in V$에 대하여 $a(x+y)=ax+ay$임. (분배법칙)\\
(VS8) 모든 $a,b \in F$와 모든 $x \in V$에 대하여 $(a+b)x=ax+bx$임. (분배법칙)
\end{DEF}

\subsubsection*{2) 벡터공간의 표기}
체 $F$에서의 벡터공간 $V$는 $F-$벡터공간 $V$라고 표기함.

혼동의 여지가 없는 경우 체 $F$는 생략하기도 함.

\subsubsection*{3) 벡터와 스칼라}
벡터공간 $V$의 원소를 벡터, 체 $F$의 원소를 스칼라라 함.

여기서의 벡터는 벡터공간의 원소를 가리키는 일반적인 개념임.\\
지금까지 단순 화살표로 표현해 온 벡터를 포괄하는 의미.

(VS3)을 만족하는 유일한 벡터 $0$은 $V$의 영백터(zero vector)라 함.

(VS4)를 만족하는 유일한 벡터 $y$($-x$)는 덧셈에 대한 $x$의 역벡터(additive inverse)라고 함.


\newpage


\section*{2. 관련 정리}
\subsubsection*{1) 벡터 합의 소거법칙}
\textbf{Theorem 1.1}\, $x,y,z \in V$이고 $x+z=y+z$ 일 때, $x = y$ 임.

\textbf{Corollary 1}\, (VS3)을 만족하는 벡터 $0$(영백터)은 유일함.

\textbf{Corollary 2}\, (VS4)를 만족하는 벡터 $y$(역벡터)는 유일함.

\subsubsection*{2) 스칼라 곱 관련 성질\footnote{스칼라, 벡터, 영벡터 사이의 곱에 대한 정리.}}

\textbf{Theorem 1.2}\, 모든 벡터공간 $V$에 대해서 다음이 성립함.

\begin{enumerate}
    \item 모든 벡터 $x$에 대하여 $0x= \vec{0}$임.
    \item 모든 스칼라 $a$와 모든 벡터 $x$에 대하여 $(-a)x=-(ax)=a(-x)$임.
    \item 모든 스칼라 $a$에 대하여 $a \vec{0} = \vec{0}$임.\\
\end{enumerate}



\section*{3. 벡터공간의 예시\footnote{아래의 예시들은 각 성분별로(component-wise) 계산이 가능하기 때문에 매번 8가지 조건들을 검토할 필요가 없음.}}
\subsubsection*{1) $n$순서쌍(n-tuple)의 집합($F^n$)}
체 $F$에서 성분을 가져온 $n$순서쌍의 집합을 $F^n$이라 표기함.

$u=(a_1,a_2, ... ,a_n) \in F^n$, $v=(b_1,b_2, ... ,b_n) \in F^n$, $c \in F$ 일 때, 합과 스칼라 곱을 아래와 같이 정의하면 이 집합은 $F$-벡터공간임.

\[u+v=(a_1+b_1,a_2+b_2, ... ,a_n+b_n),\,cu=(ca_1,ca_2, ... ,ca_n)\]

$F^n$의 벡터는 행벡터(row vector)보다 열벡터(column vector)로 주로 표현함.

$F^1$은 그냥 $F$로 표현하는 경우가 많음.


\subsubsection*{2) 행렬(matrix)의 집합($M_{m \times n}(F)$)}
성분이 체 $F$의 원소인 모든 $m \times n$ 행렬의 집합은 $M_{m \times n}(F)$라 정의함.

$A,B \in M_{m \times n}(F)$, $c \in F$ 일 때, 합과 스칼라 곱을 아래와 같이 정의하면 이 집합은 $F$-벡터공간임.

\[(A+B)_{ij}=A_{ij}+B_{ij},\,(cA)_{ij}=cA_{ij}\]


\subsubsection*{3) 함수(function)의 집합($\mathcal{F}(S,F)$)}
체 $F$의 공집합이 아닌 부분집합 $S$가 있을 때, $\mathcal{F}(S,F)$는 $S$에서 $F$로 가는 모든 함수의 집합임.

$\mathcal{F}(S,F)$에서 모든 $s \in S$에 대하여 $f(s)=g(s)$일 때, 두 함수 $f$, $g$는 같다고 함.

$f,g \in \mathcal{F}(S,F)$, $c \in F$, $s \in S$ 일 때, 합과 스칼라 곱을 아래와 같이 정의하면 이 집합은 $F$-벡터공간임.

\[(f+g)(s)=f(s)+g(s),\,(cf)(s)=c(f(s))\]

실수집합 $R$에서 $R$로 가는 모든 연속함수의 집합을 $C(R)$이라 함.\\


\subsubsection*{4) 다항식(polynominal)의 집합($P(F)$)}
체 $F$에서 계수를 가져온 모든 다항식의 집합을 $P(F)$라 함.

두 다항식의 합과 스칼라 곱을 아래와 같이 정의하면 이 집합은 $F$-벡터공간임.

\[f(x)+g(x)=(a_n+b_n)x_n+(a_{n-1}+b_{n-1})x^{n-1}+ \cdots +(a_1+b_1)x+(a_0+b_0)\]
\[cf(x)=ca_nx_n+ca_{n-1}x^{n-1}+ \cdots +ca_1x+ca_0\]

음이 아닌 정수 $n$에 대하여 $P_n(F)$는 $n$ 이하의 차수를 가진 다항식의 집합임.


\subsubsection*{5) 수열의 집합}
체 $F$에서 정의된 모든 수열의 집합을 $V$라 할 때, $t \in F$와 두 수열 $(a_n)$, $(b_n)$에 대해서 합과 스칼라 곱을 정의하면 이 집합은 $F$-벡터공간임.

\[(a_n)+(b_n)=(a_n+b_n),\,t(a_n)=(ta_n)\]\\


\newpage


\part*{2. 부분공간(subspace)}

\section*{1. 부분공간\footnote{부분공간은 벡터공간의 일부분을 살펴보기 위해 존재한다고 이해하자.}}
\subsubsection*{1) 정의\\}
\begin{DEF}
$F$-벡터공간 $V$의 부분집합 $W$가 $V$에서 정의한 합과 스칼라 곱을 가진 $F$-벡터공간일 때, $W$를 $V$의 부분공간이라 함.
\end{DEF}

즉, 벡터공간인 부모의 연산을 그대로 물려받은 벡터공간인 부분집합.

모든 벡터공간 $V$에 대하여 $V$와 $\{ 0 \}$은 $V$의 부분공간임.\\
특히 ${0}$는 점공간인 부분공간(zero subspace)이라고 함.

\subsubsection*{2) 부분공간 판별}
어떤 부분집합이 부분공간이기 위한 필요충분조건은 아래 4가지 성질을 만족하는 것임.\\
벡터공간의 8가지 조건을 생각해 보면 당연한 이야기.

\begin{enumerate}
    \item 모든 $x \in W$, $y \in W$에 대하여 $x+y \in W$임. ($W$는 합에 대하여 닫혀 있음)
    \item 모든 $c \in F$와 모든 $x \in W$에 대하여 $cx \in W$임. ($W$는 스칼라 곱에 대하여 닫혀 있음)
    \item $W$는 영벡터를 포함함. (영벡터 존재)
    \item $W$에 속한 모든 벡터 각각의 덧셈에 대한 역벡터는 $W$의 원소임. (역벡터 존재)
\end{enumerate}

Theorem 1.3에 따르면 $W$와 $V$의 영벡터는 반드시 같고, 4번 성질은 굳이 확인할 필요가 없음(항상 성립)\\


\section*{2. 관련 정리}
\subsubsection*{1) 부분공간 판별}
\textbf{Theorem 1.3}\, 벡터공간 $V$와 그 부분집합 $W$에 대하여, $W$가 $V$의 부분공간이기 위한 필요충분조건은 아래의 세 가지 조건을 만족하는 것임.

\begin{enumerate}
    \item $0 \in W$
    \item 모든 $x \in W$, $y \in W$에 대하여 $x+y \in W$
    \item 모든 $c \in F$와 모든 $x \in W$에 대하여 $cx \in W$
\end{enumerate}

\subsubsection*{2) 부분공간의 교집합}

\textbf{Theorem 1.4}\, 벡터공간 $V$의 부분공간들의 임의의 교집합은 $V$의 부분공간임.\\


\newpage


\part*{3. 일차결합과 연립일차방정식}

\section*{1. 일차결합(linear combination)}

\subsubsection*{1) 정의\\}
\begin{DEF}
벡터공간 $V$의 공집합이 아닌 부분집합 $S$가 있다고 하자. 유한개의 벡터 $u_1,u_2, \cdots ,u_n \in S$와 스칼라 $a_1,a_2, \cdots , a_n$에 대하여 아래를 만족하는 벡터 $v \in V$를 $S$의 일차결합이라 함.

\[v=a_1u_1+a_2u_2+ \cdots +a_nu_n\]

이때, $v$는 벡터 $u_1,u_2, \cdots , u_n$의 일차결합이라 하고, $a_1,a_2, \cdots , a_n$은 이 일차결합의 계수(coefficient)라고 함.
\end{DEF}

영벡터는 공집합이 아닌 모든 부분집합의 일차결합임.\\


\section*{2. 연립일차방정식}
\subsubsection*{1) 일차결합과 연립일차방정식}
어떤 벡터가 일차결합으로 표현될 수 있는지를 연립일차방정식의 해를 구해 알아낼 수 있음.

어떤 벡터가 일차결합으로 표현될 수 있는지는 연립일차방정식의 풀이와 관련이 있는데, 여기서는 방법만 살펴보고 자세한 이야기는 3장에서 함.

\subsubsection*{2) 연립일차방정식의 풀이}
아래의 세 가지 연산을 반복하여 연립일차방정식이 아래의 성질을 가지도록 하면 미지수를 다른 미지수에 대하여 쉽게 풀 수 있음.

\textbf{연산1} \,두 방정식의 위치를 바꿈.\\
\textbf{연산2} \,방정식에 0이 아닌 상수를 곱함.\\
\textbf{연산3} \,상수를 곱하여 얻은 방정식을 다른 방정식에 더함.

\textbf{성질1} \,각 방정식에서 처음으로 등장하는 0이 아닌 계수는 1임.\\(맨 앞은 계수가 1)\\
\textbf{성질2} \,어떤 미지수가 어떤 방정식에서 처음 등장하면 그 외의 다른 행에서는 등장하지 않음.\\(맨 앞 미지수는 위아래가 비어있음)\\
\textbf{성질3} \,처음 등장하는 미지수의 첨자는 다음 행으로 내려갈 때마다 반드시 증가함.\\(맨 앞 미지수 첨자는 내려갈수록 커짐)

연산 도중 $0=c$(c는 0이 아닌 스칼라) 형태의 식이 나오면 해당 연립방정식은 해가 없음을 의미함.\\


\newpage


\section*{3. 생성공간}

\subsubsection*{1) 정의\\}
\begin{DEF}
벡터공간 $V$의 공집합이 아닌 부분집합 $S$에 대해서, $S$의 벡터를 사용하여 만든 모든 일차결합의 집합을 $S$의 생성공간(span)이라 하고, $span(S)$라 표기함.
\end{DEF}

편의를 위해 $span(\emptyset)= \{ 0 \}$으로 정의함.

\subsubsection*{2) 생성}
벡터공간 $V$의 부분집합 $S$에 대햐어 $span(S)=V$이면 $S$는 $V$를 생성한다(generate, span)고 하고, $S$를 $V$의 생성집합이라고 함.

$S$의 벡터가 $V$를 생성한다고 하기도 함.

어떤 벡터가 벡터공간을 생성하는지 확인하려면, 생성되는 벡터를 미지수로 놓고 완전히 성립하는지 보면 됨.\\

\section*{4. 관련 정리}
\subsubsection*{1) 생성공간과 부분공간 사이의 관계}
\textbf{Theorem 1.5}\, 벡터공간 $V$의 임의의 부분집합 $S$의 생성공간은 $S$를 포함하는 $V$의 부분공간임. 또한 $S$를 포함하는 $V$의 부분공간은 반드시 $S$의 생성공간을 포함함.

즉, 벡터공간의 부분집합으로 만든 생성공간(span)은 부분공간임.\\
또한 부분공간의 부분집합으로 만든 생성공간(span)은 해당 부분공간에 포함됨.\\


\newpage


\part*{4. 일차종속과 일차독립}

\section*{1. 일차종속(linearly dependent)/일차독립(linearly independent)}

\subsubsection*{1) 정의\\}
\begin{DEF}
벡터공간 $V$의 부분집합 $S$에 대하여 $a_1u_1+a_2u_2+ \cdots +a_nu_n=0$을 만족시키는 유한계의 서로 다른 벡터 $u_1,u_2, \cdots ,u_n \in S$와 적어도 하나는 0이 아닌 스칼라 $a_1,a_2, \cdots ,a_n$이 존재하면 집합 $S$는 일차종속(linearly dependent)라 함. 이때, $S$의 벡터 또한 일차종속이라 함.

벡터공간 $V$의 부분집합 $S$가 일차종속이 아니면 일차독립(linearly independent)라 함. 이때. $S$의 벡터 또한 일차독립이라 함.
\end{DEF}

\subsubsection*{2) 영벡터의 자명한 표현}
임의의 벡터 $u_1,u_2, \cdots ,u_n$에 대하여 $a_1=a_2= \cdots =a_n=0$이면 $a_1u_1+a_2u_2+ \cdots +a_nu_n=0$임. 이를 $u_1,u_2, \cdots ,u_n$의 일차결합에 대한 영벡터의 자명한 표현(trivial representation of 0)이라 함.

즉, 일차결합에서 스칼라에 전부 0을 넣어 표현한 영벡터를 의미함.


\subsubsection*{3) 일차독립인 집합에 대한 명제}
일차독립인 집합에 대한 아래의 명제들은 모든 벡터공간에서 참임.

\begin{enumerate}
    \item 공집합은 일차독립임.
    \item 영이 아닌 벡터 하나로 이루어진 집합은 일차독립임.
    \item 어떤 집합이 일차독립이기 위한 필요충분조건은 영벡터를 주어진 집합에 대한 일차결합으로 표현하는 방법이 자명한 방법 뿐인 것임.
\end{enumerate}

즉, 3번 명제를 사용하여 유한집합이 일차독립인지 확인할 수 있음.


\subsubsection*{4) 일차종속/일차독립이 가지는 의미}
일차독립\\
= 영벡터를 자명한 표현으로만 나타낼 수 있음.\\ 
= 해당 집합의 모든 벡터가 다른 벡터들의 일차결합으로 표현되지 않음.


\subsubsection*{5) 일차종속/일차독립의 판정}
1. 어떤 벡터가 다른 벡터들의 일차결합으로 표현되는지 확인.

2. 영벡터가 자명한 표현으로만 나타내지는지 확인.\\
연립일차방정식을 풀어서 확인할 수 있음.\\
3장의 행간소사다리꼴(RREF)에 대한 해석을 활용하여 확인할 수 있음.


\newpage


\section*{2. 관련 정리}
\subsubsection*{1) 일차종속/일차독립과 집합관계}

\textbf{Theorem 1.6}\, $V$는 벡터공간이고 $S_1 \subseteq S_2 \subseteq V$일 때, $S_1$이 일차종속이면 $S_2$도 일차종속임.

\textbf{Corollary 1}\, $V$는 벡터공간이고 $S_1 \subseteq S_2 \subseteq V$일 때, $S_2$가 일차독립이면 $S_1$도 일차독립임.

\subsubsection*{2) 일차독립과 벡터의 유일한 표현}

\textbf{Theorem 1.7}\, 벡터공간 $V$과 일차독립인 $V$ 부분집합 $S$가 있음. $S$에 포함되지 않는 벡터 $v \in V$에 대하여, $S \cup \{v\}$가 일차종속이기 위한 필요충분조건은 $v \in span(S)$임.

즉, 일차독립인 집합 $S$의 원소들을 일차결합하여 만들 수 있는 벡터가 $S$에 추가된 집합은 일차종속임. 반대로, 일차결합하여 만들 수 없는 벡터가 추가되면 여전히 일차독립임.

\begin{proof}
$S \cup \{ v \}$가 일차종속이면 $u_1,u_2, \cdots ,u_n \in S$와 스칼라 $a_1,a_2, \cdots ,a_{n+1}$에 대해서 $a_1u_1+a_2u_2+ \cdots +a_nu_n+a_{n+1}v=0$, $a_{n+1} \not= 0$이 성립함. 정리하면 $v = - \frac{1}{a_{n+1}}(a_1u_1+a_2u_2+ \cdots +a_nu_n+a_{n+1}v=0)$이므로 $v \in span(S)$임.
\end{proof}

\begin{proof}
$v \in span(S)$이므로 $v = a_1u_1+a_2u_2+ \cdots +a_nu_n+a_{n+1}v=0$으로 표현할 수 있고, 정리하면 $a_1u_1+a_2u_2+ \cdots +a_nu_n+(-1)v=0$으로 자명적이지 않은 표현이 존재하므로 $S \cup \{ v \}$는 일차종속임.
\end{proof}


\newpage


\part*{5. 기저와 차원}

\section*{1. 기저(basis)}

\subsubsection*{1) 정의\\}
\begin{DEF}
벡터공간 $V$와 부분집합 $\beta$에 대해서, $\beta$가 일차독립이고 $V$를 생성하면 $\beta$를 $V$의 기저(basis)라 함. $\beta$가 $V$를 형성할 때, $\beta$의 벡터는 $V$의 기저를 형성함.
\end{DEF}

즉, $V$의 기저인 $\beta$는 $V$를 생성하고 일차독립임.

기저는 유한집합이 아닐 수 있음.\footnote{ex. 집합 $\{1,x,x^2, \cdots \}$은 $P(F)$의 기저임.}


\subsubsection*{2) 표준기저}
벡터공간 $F^n$의 벡터 $e_1=(1,0, \cdots ,0),\,e_2=(0,1, \cdots ,0), \cdots ,e_n=(0, \cdots ,1)$에 대하여, 집합 $\{e_1,e_2, \cdots ,e_n\}$은 $F^n$의 표준기저임.\\
여기서의 $e_n$은 임의로 잡은 기호가 아니라, 벡터공간 $F^n$의 표준기저를 일반적으로 나타내는 기호임.

집합 $\{1,x,x^2, \cdots ,x^n\}$은 벡터공간 $P_n(F)$의 표준기저임.

행렬 $E^{ij} \in M_{m \times n}(F)$는 $i$행 $j$열 성분만 1이고, 나머지 성분은 0인 행렬임. 이를 $M_{m \times n}(F)$의 표준기저로 사용하기도 함.\\


\section*{2. 차원(dimension)}
\subsubsection*{1) 정의\\}
\begin{DEF}
기저가 유한집합인 벡터공간을 유한차원(finite dimension)이라 하고, 유한차원이 아닌 벡터공간을 무한차원(infinite dimension)이라 함. $V$의 기저가 $n$개의 벡터로 이루어질 때, 유일한 자연수 $n$은 $V$의 차원(dimension)이고, $dim(V)$라 표기함.
\end{DEF}

대체정리의 Corollary 1에서 알 수 있듯이, 기저를 형성하는 벡터의 개수는 벡터공간 $V$의 본질적 성질임.

\subsubsection*{2) 예시}
벡터공간 $\{0\}$의 차원은 0임.\\
벡터공간 $F^n$의 차원은 $n$임.\\
벡터공간 $M_(m \times n)(F)$의 차원은 $mn$임.\\
벡터공간 $P_n(F)$의 차원은 $n+1$임.\\
벡터공간 $P(F)$는 무한차원임.


\section*{3. 관련 정리}
\subsubsection*{1) 부분집합이 기저가 되기 위한 필요충분조건}
\textbf{Theorem 1.8}\, 벡터공간 $V$의 부분집합 $\beta = \{u_1,u_2, \cdots ,u_n\}$가 $V$의 기저이기 위한 필요충분조건은, 임의의 벡터 $v \in V$를 $\beta$에 속한 벡터의 일차결합으로 나타낼 수 있고 그 표현이 유일한 것임.

즉, 기저 $\beta$는 유일한 일차결합으로 $V$의 벡터\footnote{$\beta$의 벡터가 아닌 $V$의 벡터인 것 유의.}를 표현할 수 있고, $\beta$의 유일한 일차결합으로 $V$의 벡터가 표현된다면 $\beta$는 기저인 것.


\newpage


\subsubsection*{2) 생성집합과 기저의 관계\footnote{아래의 증명 방식을 눈여겨보자. 특히 일차독립인 집합을 생성하고, 그 집합이 벡터공간을 생성하는지 확인하는 방식에 집중할 것.}}
\textbf{Theorem 1.9}\, 유한집합 $S$가 벡터공간 $V$를 생성하면, $S$의 부분집합 중 $V$의 (유한집합인) 기저가 존재함.

\begin{proof}
$S=\emptyset$ 또는 $S=\{0\}$인 경우, $V=\{0\}$임. $\emptyset$은 일차독립이므로, $S$의 부분집합이면서 $V$의 기저임.

$S$가 영벡터가 아닌 벡터 $u_1$을 가지고 있다고 가정하면, $\{u_1\}$은 일차독립인 $S$의 부분집합임. 집합 $\{u_1,u_2, \cdots ,u_k\}$가 일차독립이 되도록 $S$에서 순차적으로 $u_2, \cdots ,u_k$를 꺼내서 추가하는 것을 유한 번 반복함. 최종적으로 얻은 일차독립인 집합을 $\beta = \{u_1,u_2, \cdots ,u_k\}$라 함.

$S$에서 일차독립인 부분집합을 추출했으니, 이제 이 부분집합이 $V$를 생성하는지 확인해야 함.\\
(i) $\beta = S$인 경우. $S$는 일차독립이고 $V$를 생성하므로 $V$의 기저임.\\
(ii) $\beta$가 $S$의 일차독립인 진부분집합인 경우. $span(\beta)$는 $V$의 부분공간이고, $S$를 포함하는 $V$의 부분공간은 $span(S)$(즉, $V$.) 또한 포함하므로\footnote{정리 1.5}, $S \subseteq span(\beta)$인지 증명하면 $\beta$는 $V$를 생성한다고 할 수 있음.

$v \in S$에 대하여 $v \in \beta$이면 당연히 $v \in span(\beta)$임. $v \notin \beta$이면 $\beta \cap \{v\}$는 $\beta$의 구성 방식에 의해 일차종속임. $\beta \cap \{v\}$가 일차종속이므로 $v \in span(\beta)$임\footnote{정리 1.7}. 따라서 $S \subseteq span(\beta)$임.
\end{proof}

\subsubsection*{3) 대체정리(replacement theorem)\footnote{대체정리는 수학적 귀납법으로 증명하지만 그 내용은 정리하지 않음. 여기서는 Corollary 1에 집중하자.}}
\textbf{Theorem 1.10}\, $n$개의 벡터로 이루어진 집합 $G$가 벡터공간 $V$를 생성한다고 하자. $L$이 $m$개의 벡터로 이루어진 일차독립인 $V$의 부분집합이면, $m \leq n$임. 또한 다음 조건을 만족시키는 $H \subseteq G$가 존재함. $H$는 $n-m$개의 벡터로 이루어졌으며, $L \cup H$는 $V$를 생성함.

\textbf{Corollary 1}\, 벡터공간 $V$가 유한집합인 기저를 포함한다고 가정하면, $V$의 모든 기저는 유한집합이며 같은 개수의 벡터로 이루어져 있음.

\begin{proof}
$\beta$가 $n$개의 벡터로 이루어진 $V$의 기저이고, $\gamma$가 또 다른 $V$의 기저라고 하자. $\gamma$가 $n+1$개의 벡터로 이루어져 있다고 하면, $\gamma$는 일차독립이고 $\beta$는 $V$를 생성하므로 대체정리에 의해 $n+1 < n$이 성립해야 하는데 이는 모순임. 즉, $\gamma$가 $m$개의 벡터로 이루어져 있다면 $m \leq n$임. $\beta$와 $\gamma$를 바꾸어 똑같은 논리를 반복하면 $n \leq m$이므로 $m=n$임. 즉, $V$의 모든 기저는 같은 개수의 벡터로 이루어져 있음.
\end{proof}

\textbf{Corollary 2}\, $V$를 차원이 $n$인 벡터공간이라 하자.\\
(1) $V$의 유한 생성집합에는 반드시 $n$개 이상의 벡터가 있음. 또한 $n$개의 벡터로 이루어진 $V$의 생성집합은 $V$의 기저임.\\
(2) 일차독립이고 $n$개의 벡터로 이루어진 $V$의 부분집합은 $V$의 기저임.\\
(3) 집합 $L \subset V$가 일차독립이면 $L \subseteq \beta$인 기저 $\beta$가 존재함. 즉, 일차독립인 $V$의 부분집합을 확장시켜 기저를 만들 수 있음.

3번 정리를 Basis Extension Theorem이라고 함. 일차독립인 집합으로 원하는 기저를 생성할 수 있다는 의미.

3장에서 등장하는 행간소사다리꼴(RREF)의 해석을 사용하면 일차독립인 집합으로 기저를 실제로 생성할 수 있음.

정리하면, 유한차원 벡터공간 $V$에서 $dim(V)=n$이라 할 때, 일차독립인 부분집합은 벡터의 개수가 $n$보다 클 수 없고, $V$를 생성하는 부분집합은 $n$보다 작을 수 없음. 기저는 일차독립인 집합의 집합과 생성집합의 집합의 교집합으로, 그 크기가 $n$임.


\newpage


\subsubsection*{4) 부분공간의 차원}
\textbf{Theorem 1.11}\, 유한차원 벡터공간 $V$에 대하여 부분공간 $W$는 유한차원이고, $dim(W) \leq dim(V)$임. 특히, $dim(W) = dim(V)$이면 $V=W$임.

\begin{proof}
$dim(V)=n$이라 하자. $W=\{0\}$이면 $W$는 유한차원이고 $dim(W)=0 \leq n$임. $W \neq \{0\}$ 인 경우, $W$는 영벡터가 아닌 벡터 $x_1$을 가지고, $\{x_1\}$은 일차독립임. $\{x_1,x_2, \cdots x_k\}$이 일차독립이 되도록 $W$에서 $x_1,x_2, \cdots ,x_k$를 꺼내자. $V$의 일차독립인 부분집합은 $n$개 이상의 벡터를 가질 수 없으므로, 이 과정은 $k \leq n$인 범위 안에서 끝남. 이때, $\{x_1,x_2, \cdots x_k\}$는 일차독립이고 $W$에서 벡터를 더 추가하면 일차종속이 되고, 해당 벡터는 $span(\{x_1,x_2, \cdots x_k\})$의 원소임. 즉, $\{x_1,x_2, \cdots x_k\}$는 $W$의 기저이고, $dim(W)=k \leq n$임.

$dim(W)=n$이면, $W$의 기저는 $n$개의 벡터로 이루어지고 일차독립인 $V$의 부분집합임. 대체정리의 Corollary 2에 의해 이 집합은 $V$의 기저임. 즉, $W=V$임.
\end{proof}

\textbf{Corollary 1}\, 유한차원 벡터공간 $V$의 부분공간 $W$에 대해서, $W$의 임의의 기저를 확장하여 $V$의 기저를 얻을 수 있음.

\begin{proof}
$W$의 임의의 기저는 일차독립인 $V$의 부분집합이므로, 대체정리의 Corollary 2에 의해 이를 확장시켜 $V$의 기저를 만들 수 있음.
\end{proof}


\newpage
